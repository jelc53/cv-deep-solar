% CVPR 2022 Paper Template
% based on the CVPR template provided by Ming-Ming Cheng (https://github.com/MCG-NKU/CVPR_Template)
% modified and extended by Stefan Roth (stefan.roth@NOSPAMtu-darmstadt.de)

\documentclass[10pt,twocolumn,letterpaper]{article}

%%%%%%%%% PAPER TYPE  - PLEASE UPDATE FOR FINAL VERSION
%\usepackage[review]{cvpr}      % To produce the REVIEW version
\usepackage{cvpr}              % To produce the CAMERA-READY version
%\usepackage[pagenumbers]{cvpr} % To force page numbers, e.g. for an arXiv version

% Include other packages here, before hyperref.
\usepackage{graphicx}
\usepackage{amsmath}
\usepackage{amssymb}
\usepackage{booktabs}


% It is strongly recommended to use hyperref, especially for the review version.
% hyperref with option pagebackref eases the reviewers' job.
% Please disable hyperref *only* if you encounter grave issues, e.g. with the
% file validation for the camera-ready version.
%
% If you comment hyperref and then uncomment it, you should delete
% ReviewTempalte.aux before re-running LaTeX.
% (Or just hit 'q' on the first LaTeX run, let it finish, and you
%  should be clear).
\usepackage[pagebackref,breaklinks,colorlinks]{hyperref}


% Support for easy cross-referencing
\usepackage[capitalize]{cleveref}
\crefname{section}{Sec.}{Secs.}
\Crefname{section}{Section}{Sections}
\Crefname{table}{Table}{Tables}
\crefname{table}{Tab.}{Tabs.}


%%%%%%%%% PAPER ID  - PLEASE UPDATE
\def\cvprPaperID{Milestone 1} % *** Enter the CVPR Paper ID here
\def\confName{Stanford CS231N}
\def\confYear{2023}


\begin{document}

%%%%%%%%% TITLE - PLEASE UPDATE
\title{\\CS231N 2023 Milestone 1: Improving a Rooftop Solar Panel Segmentation Model's Robustness to Domain Shifts}

% For a paper whose authors are all at the same institution,
% omit the following lines up until the closing ``}''.
% Additional authors and addresses can be added with ``\and'',
% just like the second author.
% To save space, use either the email address or home page, not both

\author{Kerrie Wu \\
  Stanford University \\
  {\tt\small kerriewu@stanford.edu} \\
  % Examples of more authors
   \And
    Julian Cooper \\
  % ICME \\
  Stanford University \\
  \tt\small {jelc@stanford.edu} \\
   \And
   Andrea van den Haak \\
  % ICME \\
  Stanford University \\
  \tt\small {vandenhaak@stanford.edu} \\
}
\maketitle

% %%%%%%%%% ABSTRACT
% \begin{abstract}
% \end{abstract}

%%%%%%%%% BODY TEXT
\section{Introduction}
\label{sec:intro}

We are interested in segmenting and classifying rooftop solar panels from satellite images, with a focus on residential areas. In many countries (including the United States), the rollout of rooftop solar has been ad hoc and largely untracked. Besides industrial sites and city buildings, electricity utilities are largely blind to where and how much solar neighborhoods have accumulated. Being able to estimate this (and its rate of increase) allows utilities to predict power output reasonably well (from meteorology forecasts) and adjust grid management and investment. 

\section{Problem Statement}
Stanford's DeepSolar team have largely solved this challenge for the United States \cite{DeepSolar1} \cite{DeepSolar2}. However there are two major ways that the model can be improved. The original DeepSolar model has two branches, a classification branch that detects if there is a solar panel in an image, and a segmentation branch that produces a class activation map based on the image, which can be used to estimate the total area of solar panels in the image if the classification branch predicts positive \cite{DeepSolar1}. The output class activation map is smaller than the input image size in the model, which means the segmentation is approximate and limited in detail. If the segmentation is more accurate and at the same resolution as the input, it could be used to predict the type of solar panel, and generate a more accurate estimate of the solar panel area. Therefore we seek to improve the resolution of the segmentation branch in our project.

Secondly, the DeepSolar model is not robust to "distribution shifts". This means that the model accuracy drops when tested on datasets from other regions with different satellite imagery to its USA-based training data. With the help of the DeepSolar team, we want to investigate different techniques for modifying the DeepSolar segmentation model in order to improve its robustness to distribution shifts. This will improve the model's generalizability for use in countries other than the United States while requiring less (or ideally none) labeled data from each other countries for finetuning the model.

%-------------------------------------------------------------------------

\section{Literature Review}
 
We are looking into augmenting the segmentation branch of the model to produce more accurate segmentation outputs by incorporating Facebook's Segment Anything Model \cite{kirillov2023segment}. The Segment Anything Model is a foundational transformer vision model capable of zero-shot image segmentation, given an input image and point prompts of portions of the image to include, or disclude, in the mask. We anticipate that due to its extensive pretraining, we can use it to produce accurate segmentation masks for solar PV arrays that are the same size as the original input image without any finetuning.

In addition to incorporating the Segment Anything Model, we have also read several papers on techniques for measuring robustness to and handling distribution shift. 

\begin{itemize}
    \item Taori, 2020 \cite{Taori2020}: Major contributions included defining effective robustness and relative robustness metrics. The authors also concluded that robustness to synthetic data distribution shifts do not imply robustness to natural distribution shifts, and that a more diverse training set improved robustness.
    \item Yao, 2022 \cite{yao2022improving}: Describes a data augmentation method called LISA adds interpolations between original input-output training example pairs to the training data. The interpolation pairs can be selected in a targeted manner to improve robustness to domain shifts.
    \item Volpi, 2018 \cite{volpi2018generalizing}: Describes how to adversarially augment the training data to achieve a more robust model. At each training iteration, the training examples fed to the model are augmented with examples that are considered difficult for the current model.
\end{itemize}

\section{Data Set}

We are using both the original Deepsolar US dataset containing labeled images and segmentation masks of rooftop solar photovoltaic arrays from \cite{DeepSolar1}, and a separate crowdsourced dataset of aerial images with labeled solar photovoltaic arrays from France for improving robustness to distribution shifts \cite{Kasmi2023}.  The current Deepsolar model has been trained using US solar satellite imagery. (Insert dataset statistics). The 

\section{Technical Approach}

To incorporate the Segment Anything Model \cite{kirillov2023segment}, we adopt a similar data processing pipeline as the original DeepSolar Team \cite{DeepSolar1}. We feed an input image to the original DeepSolar classification/segmentation model. If the DeepSolar model's classification branch predicts positive, then we use the segmentation branch's output class activation map (CAM) to generate point prompts to SAM: three positive (to include in the mask), and three negative (to disclude from the mask). As an initial approach, we normalize the CAM map to values between 0 and 1, and randomly select three points where the CAM value is above ___ as positive points, and three points where the CAM value is below ____ as negative points. To evaluate performance and compare to baseline, we will use intersection over union and percent error in estimated solar PV area. Preliminary results are shown in Table 1. Some example segmentations are shown in Figure 1.

The technical approach we will use evaluates quantitative metrics such as effective robustness and relative robustness (as defined by \cite{Taori2020}) to measure the effectiveness of our approaches, and percent error in predicted solar PV area to measure segmentation model performance irrespective of robustness.  Additionally, some of the approaches that we plan to try are interpolation-based data augmentation through LISA \cite{yao2022improving}, regularized fine-tuning \cite{li2021}, adversarial data augmentation \cite{volpi2018generalizing}, and test-time/training adaptation methods involving auxiliary models such as RefSeq \cite{huang2022online}. We expect that our experiments will yield plots comparing robustness metrics and percent error in estimated solar PV area across different approaches. Preliminary baseline results are shown in Table 1.

\section{Intermediate/Preliminary Results}
\begin{tabular}{||c | c | c | c | c | c | c | c  | c ||}
\hline
Model & \multicolumn{4}{c|}{US dataset} & \multicolumn{4}{c|}{French dataset} \\ [0.5ex] 
\hline\hline
{} & precision & recall & area \% error & IOU & precision & recall & area \% error & IOU \\ 
\hline
Deepsolar US (baseline) & 0.95 & 0.86 & -8.0\% & NA & 1.0 & 0.001 & 0.47\% & NA \\ 
\hline
Deepsolar US + SAM &  0.95 & 0.86 & 47.0 \% & 0.24 & TBD & TBD & TBD & TBD\\ 
\hline

\end{tabular}
\begin{center}

 Table 1: Preliminary results showing statistics of the baseline model (Deepsolar US) on the original US dataset, compared to results with the same model on the French dataset. We also compare results to those of the original Deepsolar US model but using Facebook's Segment Anything Model for segmentation. Precision and Recall are both based on the image classification performance of the model (which are the same between the baseline and with SAM). Percent area error measures the percent difference between the predicted solar PV area and the true solar PV area as measured by the true segmentation masks. IOU stands for intersection over union and is calculated only for models using SAM because the CAM maps produced by the original Deepsolar model are lower resolution than the true segmentation labels, and therefore can't be used to calculate an IOU score. 


\end{center}

\section{Next steps}
For improving usage of SAM, we can investigate different methods of picking point prompts. This can include changing the number of point prompts provided to SAM, and different selection methods (eg, tuning the CAM value thresholds, or sampling based on a probability distribution instead of randomly). Once we have a DeepSolar classification/segmentation model that performs well on the French dataset, we can also attempt to use the SAM model on the French dataset. 

(distribution shift next steps)


%%%%%%%%% REFERENCES
{\small
\bibliographystyle{ieee_fullname}
\bibliography{egbib}
}

\end{document}
