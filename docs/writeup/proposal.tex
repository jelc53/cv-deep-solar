\documentclass{article}

\usepackage[final]{neurips_2019}

\usepackage[utf8]{inputenc}
\usepackage[T1]{fontenc}
\usepackage{hyperref}
\usepackage{url}
\usepackage{booktabs}
\usepackage{amsfonts}
\usepackage{nicefrac}
\usepackage{microtype}
\usepackage{graphicx}
\usepackage{xcolor}
\usepackage{lipsum}
\usepackage{amsmath}
\usepackage[left=1.5cm, right=1.5cm]{geometry}

\newcommand{\note}[1]{\textcolor{blue}{{#1}}}

\title{
  [Walk Less and Only Down Smooth Valleys] \\
  \vspace{0.15cm}
  {\normalfont Stanford CS231N Project}  % Select one and delete the other
}

\author{
  Kerrie Wu \\
  % ICME \\
  Stanford University \\
  \texttt{kerriewu@stanford.edu} \\
  % Examples of more authors
   \And
    Julian Cooper \\
  % ICME \\
  Stanford University \\
  \texttt{jelc@stanford.edu} \\
   \And
   Andrea Van Den Haak \\
  % ICME \\
  Stanford University \\
  \texttt{vandenhaak@stanford.edu} \\
}

\begin{document}

\maketitle

% \begin{abstract}
%   Required for final report
% \end{abstract}

% \note{This template is built on NeurIPS 2019 template\footnote{\url{https://www.overleaf.com/latex/templates/neurips-2019/tprktwxmqmgk}} and provided for your convenience.}
% \vspace{-0.7cm}

% \newline
\textbf{Problem \& Motivation.}
We are interested in classifying rooftop solar panel from satellite images, with a focus on residential areas. In many countries (including the United States) the rollout of rooftop solar has been ad hoc and largely untracked. Besides industrial sites and city buildings, electricity utilities are largely blind to where and how much solar neighborhoods have accumulated. Being able to estimate this (and its rate of increase) allows utilities to predict power output reasonably well (from meteorology forecasts) and adjust grid management and investment.

While Stanford's DeepSolar team have largely solved this challenge for the United States \cite{DeepSolar1} \cite{DeepSolar2}, their model is not robust to "distribution shifts". This means that the model accuracy drops when tested on datasets from other regions with different satellite imagery to its USA-based training data. With the help of the DeepSolar team, we want to investigate different techniques for modifying the DeepSolar model in order to improve its robustness to distribution shifts. \\

\textbf{Dataset \& Evaluation.}
In the interest of helping to solve this specific problem for rooftop solar prediction, Kasmi et al \cite{Kasmi2023} recently published a labelled dataset for France (released January 2023). The dataset provides ground truth installation masks for 13303 images from Google Earth25 and 7686 images from the French national institute of geographical and forestry information (IGN). We can use to test our model's ability to handle distribution shifts. The idea being to continue only exposing our model United States data for training and do validation and testing on the France data. \\

\textbf{Literature Review.}
In addition to reviewing the France dataset and DeepSolar codebase, we have also read several papers on techniques for handling distribution shift.

\begin{itemize}
    \item Taori, 2020 \cite{Taori2020}: xx
    \item Kulinski, 2022 \cite{Kulinski2022}: xx
\end{itemize}

\textbf{Proposed Modeling Approach.} 
re-weighting training data, regularization, synthetic data generation, etc.



\bibliographystyle{unsrt}
\bibliography{references}



\end{document}
